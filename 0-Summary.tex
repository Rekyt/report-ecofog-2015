\section*{Summary}

Functional Ecology aims to understand Ecology through functional traits. Those functional traits correlate with performance indexes. Radial growth is an easily measurable performance index, therefore abundant literature describes its evolution. Several statistical models has been made to predict radial growth using functional traits.

Several authors pointed out the importance of considering intra-specific variability in models to gain predictive performance. However, most growth models do not include intra-specific variability explicitly.

Here we focused on 9 1ha plots in French Guiana comprising a total of 3483 trees, having both traits and growth measurements. We contrasted the intra- and inter-specific variabilities of traits and growth and found that only 30\% of the variance of growth was explained by the species effect. From what has been showed for competition we investigated how the relative position of an individual compared to its species average trait could affect performance. We compared growth model including species average, hierarchical distance, absolute distance, and individual trait terms. Including hierarchical distance or absolute distance in our growth model did not highly improve the performances of our growth model. However, depending on the trait, growth rate patterns showed how the same magnitude of variation in individual distance or in species average trait had different consequences on growth rate. For leaf toughness, increasing individual trait increased growth rate while increasing species average trait decreased growth rate. 

Opposed effects on performance unravel trade-offs between intra- and inter-specific variabilities. For other studied traits, no such trade-offs were found indicating that other ecological mechanisms are involved in species trait equilibria.