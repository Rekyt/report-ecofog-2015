\label{sec:Intro}
\section*{Introduction}
Functional Ecology aims to study organisms in terms of functional traits \cite{NEEDED}, instead of studying species.

From Hutchinson definition of \emph{fundamental niche} \citep{hutchinson_concluding_1957}, \citet{violle_towards_2009} extended the definition to become a \emph{functional niche}—a multi-dimensional volume, also called the \emph{functional space} ; it is an n-dimensional space defined by measures on n traits. In such a space an individual is defined by all the values of its traits, each one on a distinct axis;  a species niche would then be the average of trait values of all individuals of the species. In three dimensions, the species niche would be the volume encompassing all its individuals; the center of gravity of the cloud would be the species average trait values, defining niche position, while the shape of the general "cloud" would define niche breadth, as suggested by \citet{violle_towards_2009}.

The concept of functional niche helps understand community ecology and assemblage. In this view, two species would coexist if their functional volumes would not intersect. The high number of dimensions, i.e. traits, of the functional "hyper-volume" as called by \cite{NEEDED} makes it difficult to apprehend: the volume may have "holes" where some trait combinations are impossible. An individual is located by its own trait values or relatively to its species average. The distance from species average translates the intra-specific variability in the species.

Species functional niche reflects their ecological strategies. For plants, four traits have been identified to underline distinct strategies: the classical Leaf Area - Height - Seed mass triangle, suggested by \cite{NEEDED}.

We assume that species niche do not evolve dramatically over time, otherwise we would expect species to change their traits continuously. As there is no such thing as "darwinian demon" that would be perfectly fit to its environment \cite{NEEDED}. Resource and energy trade-offs limit the capacity of species to evolve \cite{NEEDED}. In some cases, clear resource and energy trade-offs has been shown \cite{NEEDED}, for example, if a tree makes denser wood it has increased resistance to herbivory but it grows slower \citep{baraloto_decoupled_2010}.

Those trade-offs are generally shown at the species level by comparing several species showing impossible trait combinations, i.e. in the previous example it would be impossible to find species with dense wood and fast growth. However, we here want to investigate not a general trade-off that exists at the species level, but how the position of an individual compared to the species average may influence its growth. For example, a tree with a less dense wood than its species average would have a higher growth rate than individuals with a density around the species average. Throughout this paper we try to understand how the performance of an individual is affected by the distance to its species average trait.

For certain traits, we would expect the species position not to vary, thus, our approach may unravel new trade-offs between intra-specific and inter-specific variablities.

Diameter Growth \& Tropical Forests $\rightarrow$ French Guiana Context

Several growth models created, used $\rightarrow$ estimate growth using measured traits

In tree growth model, intra-specific variability is rarely considered, i.e. all individuals of the same species share the average species trait value.

However, there is trait variability, intra-specific variability. What importance does it have? To what extent is it important to consider it? 

Being different from mean species trait $\rightarrow$ importance?

