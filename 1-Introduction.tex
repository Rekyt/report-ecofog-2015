\label{sec:Intro}
\section*{Introduction}

Functional ecology aims to understand ecology through the core concept of functional trait~\citep{mcgill_rebuilding_2006}. Functional traits are measurable properties of organisms that strongly influence organismal performance~\citep{mcgill_rebuilding_2006}, relating indirectly to the fitness of an individual. Performance can be defined as a measurable quantity enabling comparison between locations, species or individuals on their capacity to maintain biomass or the gene pool over generations. Focusing on functional traits also allowed to expand the historical concept of \emph{ecological niche} from~\citet{hutchinson_concluding_1957}, to the concept of \emph{functional niche}~\citep{violle_towards_2009}. It extended community ecology and sharpened understanding of community assemblages~\citep{kraft_functional_2010}. Several hypotheses were suggested to predict assemblages: competitive exclusion implies that more closely related individuals in traits experience greater competition than more distant ones, environmental filtering suggest that abiotic conditions (i.e. temperature, precipitation, etc.) select for certain types of traits in the community,~\citet{kunstler_competitive_2012} instead suggested that the hierarchical trait difference between trees drove community assemblage in alpine forests.

Community dynamics, i.e. the change of community composition through time, can be studied through performance indexes. They should be comparable across species and environmental gradients to explain the variety of processes going on. It was assumed as underlined by~\citet{mcgill_rebuilding_2006} that population increase rate was the best possible measure. In their review they argued that performance indexes should be easily measurable on a great number of species	and connected to physiology, such as seed output or tree height, that reflect reproductive and light acquisition strategies respectively. Those performance indexes are related to population dynamics and they integrate various facets of them: growth vs. survival for example.

Radial growth is an example of performance currency for trees, it has been used in wide range of environmental gradients~\citep{herault_functional_2011, kunstler_competitive_2012}. As radial growth is related to biomass production, it is a key parameter to understand the potential $\text{CO}_2$ sink that forests represent worldwide.~\citet{herault_functional_2011} modeled radial growth from functional traits, making a highly generic model adapted to the various shapes that radial growth can take through ontogeny across tropical tree species. However, in their model, they did not considered intra-specific variability of traits, i.e. that the individuals of a given species do not share the same traits; instead, in the model, individuals of a species have all the same traits equal to the species average value — intra-specific variability is not modeled explicitly.

Growing literature in functional ecology underline the importance of taking intra-specific variability into account both in trait values and performance in models~\citep{violle_towards_2009, clark_high-dimensional_2010, albert_when_2011, violle_return_2012}.~\citet{albert_when_2011} still point out that intra-specific variability can be ignored if negligible compared to inter-specific variability. If not they underline that intra-specific variability in traits may have consequences on performance indexes. For example, a tree with a denser wood than its species average density may have a lower growth than its species average growth. \citet{kunstler_competitive_2012} showed how neighbors tree one of species $A$ and the other one of species $B$, with trait $t_A$ and $t_B$ respectively, had their performances driven by their trait hierarchy $t_A - t_B$. For most traits, they showed that hierarchical distance $(t_A - t_B)$ better predicts performance than absolute traits distance $\vert t_A - t_B \vert$. Instead of making a hierarchy between to neighbors trees, we could position the traits of an individual tree compared to its species average value to unravel the interplay between inter-specific (the species' position) and intra-specific (the individual's position) variabilities and how they affect performance.

Using radial growth and traits data sets of 9 1ha plots spread over French Guiana~\citep{baraloto_decoupled_2010}~(\autoref{fig:map}) we wandered (i) how intra-specific variabilities are structured both in traits and in performance compared to inter-specific variabilities. (ii) Are those intra-specific variabilities due to the environment? (iii) Are intra- and inter- specific variaiblities related, i.e. does a tree with traits very different from its species average has a growth very different from its species average? (iv) Does intra-specific variability in traits needs to be taken into account in growth models?