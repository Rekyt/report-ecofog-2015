\label{sec:Intro}
\section*{Introduction}

Functional Ecology $\rightarrow$ Functional Traits


From Hutchinson definition of niche \citep{hutchinson_concluding_1957}, \citet{violle_towards_2009} extended the definition to a multi-dimensional volume, called the "functional space". In such a space an individual is defined by all the values of its traits, each one on a distinct axis;  a species functional space would then be the average of trait values of all individuals of the species.



Importance of all axes to have good view of ecological strategies

Diameter Growth \& Tropical Forests $\rightarrow$ French Guiana Context

Several growth models created, used $\rightarrow$ estimate growth using measured traits

Generally consider traits as specific, all individuals of the same species sharing an average trait

However, there is trait variability, intra-specific variability. What importance does it have? To what extent is it important to consider it? 

Being different from mean species trait $\rightarrow$ importance?

