\label{sec:Intro}
\section*{Introduction}
Functional ecology aims to understand ecology not using the historical phylogenetic perspective, but through the core concept of functional trait to unravel the different functions at different scales~\citep{mcgill_rebuilding_2006}. Functional traits are measurable properties of organisms that strongly influence organismal performance~\citep{mcgill_rebuilding_2006}, relating indirectly to the fitness of an individual. Then, the traits of an individual may be a proxy for its fitness.

In Functional Ecology, numerous papers underline the importance of taking intra-specific variability into account~\cite{NEEDED}. It seems a logical next step to understand better the processes than to assume that all individuals of a given species behave the same way. Indeed, traits intra-specific variability gave us insights in the variance of processes.

French Guiana is in the North of Brazil is mostly covered by forest and it contains a part of the Amazonian forest. We focused here on 9 plots spread over French Guiana in different specific environments. All of them are in humid tropical forest.... Several performance indexes have been used such as the number of seeds, tree height or tree diameter. The density of trees in tropical forests makes it very difficult to measure height and the number of seeds of a given tree, tree diameter is easier to measure, it only needs to have access to the stem of the tree, making tree diameter the measure of choice. Several traits were measured on the same plots in 2007 \citep{baraloto_decoupled_2010}. Because all measured trees were botanically determined, we can study the inter-specific as well as the intra-specific variability both in performance and in traits.

As underlined above, intra-specific variability holds some information about the specific niche occupied by an individual, however, growth models still do not include it and consider all trees as behaving the same. Thus we wandered (i) how intra-specific variabilities was structured both in traits and in performance compared to inter-specific variabilities, (ii) what the origin of those intra-specific variabilities were, (iii) if they were related, i.e. does a tree we traits very different from its species average has a growth very different from its species average, (iv) does intra-specific variability in traits needs to be taken into account in growth models, (v) are there specific patterns of performance taking each trait separately looking at individual's position from species average trait.