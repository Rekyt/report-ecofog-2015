\label{sec:Intro}
\section*{Introduction}
Functional ecology aims to understand ecology with a different approach from the historical phylogenetic perspective, instead through the core concept of functional trait to unravel the different functions at different scales~\citep{mcgill_rebuilding_2006}. Functional traits are measurable properties of organisms that strongly influence organismal performance~\citep{mcgill_rebuilding_2006}, relating indirectly to the fitness of an individual. Performance can be defined as a measurable quantity enabling comparison between species on their capacity to maitain biomass over generations.

In Functional Ecology, numerous papers underline the importance of taking intra-specific variability into account~\citep{violle_towards_2009, albert_when_2011}. Intra-specific variability is the variability existing between individuals of the same species, compared to inter-specific variability which is the variability between different sets of species. Growth model developed by~\citet{herault_functional_2011} for example consider only inter-specific variability, the model assume that all individuals of a given species share the same traits: the average traits of all individuals of the species. Several studies reviewed by~\citet{violle_return_2012} underline the importance of intra-specific variability for our understanding of species coexistence for example; a tree with traits different from its species average may occupy a slightly different ecological niche and thus coexist in some cases, while models considering only inter-specific variability would not predict such coexistence~\citep{clark_high-dimensional_2010, paine_functional_2011}.

As underlined above, intra-specific variability holds some information about the specific niche occupied by an individual, however, growth models still do not include it and consider all trees as behaving the same. Thus we wandered (i) how intra-specific variabilities was structured both in traits and in performance compared to inter-specific variabilities, (ii) what the origin of those intra-specific variabilities were, (iii) if they were related, i.e. does a tree we traits very different from its species average has a growth very different from its species average, (iv) does intra-specific variability in traits needs to be taken into account in growth models, (v) are there specific patterns of performance taking each trait separately looking at individual's position from species average trait.

In the context Amazonian forest, which is a Neotropical forest, performance indexes may be complex to measures. Widely used indexes include tree height, seed number, flower number or tree diameter, because of vegetation density tree diameter is the only reliable measurement. We focused here on 9 1-ha plots \missfig spread over French Guiana along precipitation and geological gradient, where 16 traits were measured~\citep{baraloto_decoupled_2010}, and tree diameter was followed for several decades, comprising a total of 3483 trees. Because all measured trees were botanically determined, we can study the inter-specific as well as the intra-specific variability both in performance and in traits.