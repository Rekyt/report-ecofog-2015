\section*{Materials and Methods}
\label{sec:M&M}

\subsection*{Data Provenance}

\subsubsection*{Growth Data}
The first data set is an inventory of all trees over 10cm in Diameter at Breast Height (DBH), i.e. measured at 1.3m high, in nine 1-ha plots in French Guiana (see map \missfig). In each plot, trees diameter were measured every two or five years depending on the plot.

We selected a common measured period between 2001 and 2013 comprising a total of 3549 trees; we estimated annual growth rate (AGR) in diameter by fitting a linear regression of DBH over years. The slope of the regression gave us an average AGR for each followed tree on the comprised 


\subsubsection*{Trait Data}

The second data set comes was a collection of five functional traits (see~\autoref{tab:seltraits}) extracted from a bigger database \citep{baraloto_decoupled_2010} on the same trees. Traits were not followed through time and measured only once. Selected traits are related to leaf and wood economics spectrum\citep{westoby_leaf-height-seed_1998, baraloto_decoupled_2010}.

\textbf{Leaf economics spectrum.} Specific Leaf Area (SLA) is the photo-sensitive area per unit of dry mass of the leaf; high SLA underlines investment on high light-capturing leaves that have a short payback time per gram of dry matter invested; while low SLA reflects strategies with less light-capturing leaves and longer payback time that may appear competitive in some conditions. Total leaf chlorophyll content reflects the global strategy of the plant of having resource-expansive leaves with high payback or resource-cheap leaves with lower payback. Laminar toughness measures the resistance of a leaf to pinching, high toughness values correlates with low herbivory rate, it correlates with defense strategy~\citep{westoby_leaf-height-seed_1998}.

\textbf{Stem economics spectrum.} Wood density underlines different ecological strategy for trees, a low wood density makes wood less stable and less better protected against herbivory but cheap volumetric construction cost because of low resource requirements; while a high wood density makes the tree more stable but with higher construction cost, meaning a lower growth. Trunk bark thickness associate with defense strategies in neotropical forests, thicker bark provides higher resistance to pathogens and herbivores~\citep{paine_functional_2010}.

\subsection*{Analysis of Variance}
Variance partitioning was done using a one-way Analysis of Variance, we explained either individual traits by a species effect and an individual term error, as follows:
\begin{equation}
	\label{eq:anova}
	\text{Tr}_{i,s} = \mu_s + \epsilon_i,
\end{equation}
with $\text{Tr}_{i,s}$, the trait of individual $i$ of species $s$; $\mu_s$ the mean trait of species $s$; $\epsilon_i$ the individual error term with a Gaussian distribution. The explained variance by the species effect can then be expressed by the proportion of group sum of squares over the total sum of squares. We considered the residual sum of squares as being the individual variance. We partitioned the variance similarly for AGR.

\subsection*{Growth model}

To predict the growth of tree from a single trait we used a linear mixed-model with the general formula:
\begin{equation}
	\label{eq:growth_mod}
	\log(\text{AGR}_{i, s, p} + 1) = \theta_0 + \gamma_{0, s} + \gamma_p
		+ (\theta_1 + \gamma_{1, s}) \times \text{DBH}
		+ (\theta_2 + \gamma_{2, s}) \times \log(\text{DBH})
		+ \delta
		+ \epsilon_i,
\end{equation}
with $\epsilon_i \sim \mathcal{N}(0, \theta_3)$ the individual residual,
where $\text{AGR}_{i, s, p}$ is the AGR of tree $i$ of species $s$ in plot $p$; $\theta_0 \ldots \theta_3$ are parameters to be estimated; $\gamma_{0, s} \ldots \gamma_{2, s}$ and $\gamma_p$ follow a centered Gaussian distribution with unknown variances $\sigma^2_{0, s} \ldots \sigma^2_{2, s}$ and $\sigma^2_p$. $\text{Tr}_s$ is the average trait value for species $s$.

To understand how the distance to species average value affected the predicted growth, we used different $\delta$ values:
\begin{equation}
	\label{eq:delta}
	\delta = \left\{
	\begin{aligned}
	\theta_4 \times \text{Tr}_s& \\
	\theta_4 \times \text{Tr}_s& + \theta_5 \times(\text{Tr}_i - \text{Tr}_s) \\
	\theta_4 \times \text{Tr}_s& + \theta_5' \times \vert \text{Tr}_i - \text{Tr}_s \vert \\
	\theta_4' \times \text{Tr}_i \\
	\end{aligned}
	\right.
\end{equation}

 $\text{Tr}_s$ the species average value, with a parameter $\theta_4$; $\text{Tr}_i - \text{Tr}_s$ the distance of individual trait value $\text{Tr}_i$ to species average, with the species term $\text{Tr}_s$; or $\vert \text{Tr}_i - \text{Tr}_s \vert$ the absolute distance to species average trait, with the species term $\text{Tr}_s$; or $\text{Tr}_i$ the individual trait value.

Our models tested the difference of prediction between using only the species average trait value to predict growth and the same term plus an individual distance term (real or absolute) vs. the individual trait value.

\subsection*{Simulations}

For each trait, we selected the best growth model with both the highest adapted R-squared for mixed models \citep{nakagawa_general_2013} and lowest Akaike Information Criterion (AIC); between models with either individual distance to species trait average or the absolute value of this distance (see~\autoref{tab:seltraits}). We simulated regularly spaced values of both the species average trait value and the distance to species average, within the 5$^{\text{th}}$ and 95$^{\text{th}}$ centiles of our data. From those simulated values, with fixed DBH value, we used growth models to predict AGR, depending on both species average and distance to species average trait value.

\subsection*{Data analysis}

All data analyses were made using \texttt{R} \citep{R_language} version 3.2.0 (2015-04-16), plots were made with \texttt{ggplot2} \citep{ggplot2_pkg}. We fit mixed-models with \texttt{lme4} \texttt{R} package \citep{lme4_pkg} 1.1-7 and computed adapted R-squared for mixed-models \citep{nakagawa_general_2013} implemented in \texttt{MuMIn} \texttt{R} package \citep{mumin_pkg} version 1.13.4.

\begin{table}
	\begin{center}
		\rowcolors{1}{white}{lightgray}
		\begin{tabular}{lccc}
		\hline \hline
		Trait Name & Units & Role & Best Growth models \\
		\hline
		Trunk bark thickness & mm & Stem economics & Absolute Distance\\
		Xylem density (wood density) & $\text{g}.\text{cm}^{-3}$ & Stem economics & Individual Distance* \\
		Specific Leaf Area (SLA) & $\text{cm}^2.\text{g}^{-1}$ & Leaf economics & Individual Distance\\
		Laminar total chlorophyll & $\text{µm}.\text{mm}^{-2}$  & Leaf economics & Individual Distance*\\
		Laminar toughness & N & Leaf economics & Individual Distance*\\
		\hline \hline
		\end{tabular}
		\caption{\textbf{Selected functional traits.} Stem and Leaf Economics Spectrum are defined as in~\citep{baraloto_decoupled_2010}, the two axes unravel distinct ecological strategies (see~\nameref{sec:M&M} for more details). The "Best Growth Model" column shows which growth model (\autoref{eq:growth_mod}) explained best individual trait values. \textbf{*}: Individual and Absolute distances models had similar performances.} 
		\label{tab:seltraits}
	\end{center}
\end{table}
