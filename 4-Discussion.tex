\label{sec:Disc}
\section*{Discussion}

From our data set we showed that taking into account intra-specific variability does not improve much growth models; compared to the extensive sampling needed to measure individual traits. We also showed how individual trait value had implications on tree performance when compared to its species average trait. The same variation in trait value impacted performance differently if it affected intra- or inter-specific variabilities.

\subsection*{Taking intra-specific variability into account in growth model}

Taking intra-specific variability into account has been a running debate in functional ecology, it is a trade-off between the cost of sampling several individuals per species and the information gained by such sampling~\citep{albert_when_2011, violle_return_2012}. Indeed, in our data set, for radial growth, it seemed reasonable to link intra-specific growth variability to traits intra-specific variabilities; however, we have shown here that those variabilities are only slightly related. For wood density or bark thickness, the use of hierarchical distance improved the prediction by around a 10\% factor in R-squared, while for the other traits it was less than a factor 1\% in R-squared. Thus, in our case, averaging traits for entire species and consider each individual as an average of all other individuals in its species does not affect predictive performances highly.

\subsection*{Variabilities in radial growth and traits}

We showed that traits and growth intra-specific variabilities do not seem to correlate well in our case. Several causes may however explain those variabilities. Micro-environmental variations may cause each individual to exhibits variations in AGR or traits and lead to intra-specific variability. However, we saw no spatial auto-correlation in traits (data not shown) nor in growth model residuals.~\citet{albert_intraspecific_2010} suggested that genetic structure between different populations of the same species could structure traits intra-specific variability: sub-populations may be genetically more related inside themselves then between them, having traits more similar inside the same sub-population.

\subsection*{Interplay of inter-specific and intra-specific trait variabilities}

Intra- vs. inter-specific trait variabilities affect AGR patterns differently. We have shown using performance landscapes (see~\autoref{fig:simul}) that for a each trait a variation in hierarchical distance or the same variation in species average had different consequences on performance. We underlined very diverse patterns of variations depending on the trait used. However, those patterns suggest striking interpretation: for example a continuous decrease in wood density seems to highly increase AGR. Trees, still, do not decrease their wood density as such. Radial growth is a performance index related mainly to tree survival~\citep{aubry-kientz_vigor_2015} and not to fitness, it does not indicate the evolutionary path that an individual follows. Instead, radial growth gives us insight on the ecological strategy of a species, either being a fast or a slow growing species. The fact that we do not observe continuously decreasing wood density also suggest that other phenomena could limit this decrease. Wood density is a key trait that experience several trade-offs~\citep{chave_towards_2009} because of mechanical stresses or defence against herbivory that constrain its value.

Leaf toughness exhibits another contrasted pattern where increasing hierarchical distance increases AGR while increasing species average trait decreases AGR. Here intra-specific and inter-specific variabilities are in opposite direction, this may lead to a species wide trade-off for AGR that could constrain leaf toughness values. In this case,  because of opposite dynamics at the individual and species levels for a trait, the trait could reach a stable equilibrium. What may influence such an equilibrium if it exists? Is it stable over time? Species functional niches are considered stable over time, overseeing the dynamics of such spaces.


%We showed that depending on the trait, intra-specific variability may have various consequences on AGR patterns. An individual moving away from its species average bark thickness, has a lower AGR than an other individual of the same species, with the same average species value; individuals at the species average value have greater AGR. An increase in bark thickness average species value also increases performance, there is a trade-off between individual's distance and species average bark thickness values. We would imagine a species to increase its performance (=AGR) and thus increase its bark thickness average value, however, for this to happen certain individuals would have greater bark thickness than the species average value, decreasing their performance. If our performance index reflect the probable evolutionary landscape, we would then be in a situation of stable average species value. Whereas for wood density increasing the individual's distance decreases performance, and increasing species average wood density decreases AGR; from an evolutionary point of view, we would see a continuous decrease in wood density, maximizing AGR. However, we observe stable phenotypes for most species~\cite{NEEDED} on their trait, mainly because our performance index does not reflect the evolutionary landscape and fitness properly. AGR in diameter instead is a proxy of survival only, a bigger tree diameter has a higher survival rate than a smaller tree, AGR does not reflect reproduction nor seed survival. Performance indexes may be proxies of fitness, but they underline different ecological or evolutionary strategies, and care should be taken when interpreting performance landscapes.

%Our simulations still unravel new trade-offs, between individual trait and its species average trait. Abundant studies showed trade-offs between traits~\cite{NEEDED}, but, to our knowledge, it is the first time that a trade-off, for the same trait, has been shown between intra-specific variability and species average value. Taking SLA for example, individual's distance and species average trait pull in opposite directions for AGR: increasing individual's SLA increases AGR, but increasing species average SLA decreases AGR.
