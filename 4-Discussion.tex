\label{sec:Disc}
\section*{Discussion}

From our data set we showed that taking into account intra-specific variability does not improve much growth models; compared to the extensive sampling needed to measure individual traits. Using simulations, we also showed how individual's trait had implications of tree performance when compared to its species average trait.

Taking intra-specific variability into account has been a running debate in functional ecology, it is a trade-off between the cost of sampling several individuals per species and the information gained by such sampling~\citep{albert_when_2011, violle_return_2012}. Indeed, in our data set, for growth in diameter, it seemed reasonable to link intra-specific growth variability to traits intra-specific variabilities; however, we have shown here that those variabilities are not related. Thus, in our case, averaging traits for entire species, and consider each individual as an average of all other individuals in its species does not affect predictive performances.

It still raises the question of what causes those two independent variabilities. We failed to reveal a micro-environmental effect by studying spatial auto-correlation in both traits and AGR, whereas traits variation along spatial gradient has been shown in some cases~\cite{NEEDED}. Here we did not look at the influence of neighboring trees on a focal tree traits, it may drive intra-specific trait variability. The niche differentiation hypothesis for example drive trees towards trait values that distinct from their neighbors'.

We showed that depending on the trait, intra-specific variability may have various consequences on AGR patterns. An individual moving away from its species average bark thickness, has a lower AGR than an other individual of the same species, with the same average species value; individuals at the species average value have greater AGR. An increase in bark thickness average species value also increases performance, there is a trade-off between individual's distance and species average bark thickness values. We would imagine a species to increase its performance (=AGR) and thus increase its bark thickness average value, however, for this to happen certain individuals would have greater bark thickness than the species average value, decreasing their performance. If our performance index reflect the probable evolutionary landscape, we would then be in a situation of stable average species value. Whereas for wood density increasing the individual's distance decreases performance, and increasing species average wood density decreases AGR; from an evolutionary point of view, we would see a continuous decrease in wood density, maximizing AGR. However, we observe stable phenotypes for species~\cite{NEEDED} on their trait, mainly because our performance index does not reflect the evolutionary landscape and fitness properly. AGR in diameter instead is a proxy of survival only, a bigger tree diameter has a higher survival rate than a smaller tree, AGR does not reflect reproduction nor seed survival. Performance indexes may be proxies of fitness, but they underline different ecological or evolutionary strategies, and care should be taken when interpreting performance landscapes.

Our simulations still unravel new trade-offs, between individual trait and its species average trait. Abundant studies showed trade-offs between traits~\cite{NEEDED}, but, to our knowledge, it is the first time that a trade-off, for the same trait, has been shown between intra-specific variability and species average value. Taking SLA for example, individual's distance and species average trait pull in opposite directions for AGR: increasing individual's SLA increases AGR, but increasing species average SLA decreases AGR. Those trade-offs 
